\documentclass[12pt,a4paper]{article}
\usepackage[utf8]{inputenc}
\usepackage[english]{babel}
\usepackage{amsmath}
\usepackage{cite}      % 基本的な引用
\usepackage{natbib}    % 著者名-年形式の引用
\usepackage{hyperref}  % リンク機能

\title{LaTeX Citation Methods}
\author{Guide}
\date{}

\begin{document}
\maketitle

\section{Method 1: Manual Bibliography (基本)}

% 本文中での引用
According to Smith~\cite{smith2020}, the method is effective.
Multiple citations can be grouped~\cite{jones2019,brown2021}.
A specific page reference~\cite[p.~25]{smith2020}.

% 文末に参考文献リスト
\subsection*{References (Manual)}
\begin{thebibliography}{99}

\bibitem{smith2020}
J. Smith, 
``An Important Paper,''
\textit{Journal of Research}, 
vol.~15, no.~3, pp.~123--145, 2020.

\bibitem{jones2019}
A. Jones and B. Lee,
\textit{Book Title},
Publisher Name, 2019.

\bibitem{brown2021}
M. Brown,
``Web Article,''
\url{https://example.com},
Accessed: Jan. 2021.

\end{thebibliography}

\newpage
\section{Method 2: BibTeX (推奨)}

% プリアンブルに追加:
% \bibliographystyle{plain}  % スタイルを選択

% 本文中での引用(Method 1と同じ)
According to the study~\cite{doe2020}, we can see that...
Recent work~\cite{doe2020,smith2021} shows...

% 文末に(手動の代わりに):
% \bibliography{references}  % references.bib ファイルを参照

% references.bib ファイルの例:
\begin{verbatim}
@article{doe2020,
  author = {John Doe},
  title = {Research Title},
  journal = {Journal Name},
  year = {2020},
  volume = {10},
  number = {2},
  pages = {100--120}
}

@book{smith2021,
  author = {Jane Smith},
  title = {Book Title},
  publisher = {Publisher},
  year = {2021},
  address = {City}
}

@inproceedings{wang2019,
  author = {L. Wang and M. Chen},
  title = {Conference Paper},
  booktitle = {Proc. Int. Conf.},
  year = {2019},
  pages = {50--55}
}

@misc{web2021,
  author = {A. Author},
  title = {Web Page Title},
  howpublished = {\url{https://example.com}},
  year = {2021},
  note = {Accessed: Jan. 2021}
}
\end{verbatim}

\newpage
\section{Method 3: natbib (著者名-年形式)}

% プリアンブルに追加:
% \usepackage{natbib}
% \bibliographystyle{plainnat}  % or: abbrvnat, apalike

% 引用の例
\subsection{Citation Styles with natbib}

\begin{itemize}
\item \verb|\citet{key}| → Author (Year) 形式
  
  Example: \verb|\citet{smith2020}| produces: Smith (2020)

\item \verb|\citep{key}| → (Author, Year) 形式
  
  Example: \verb|\citep{smith2020}| produces: (Smith, 2020)

\item \verb|\citet*{key}| → 全著者名を表示
  
  Example: \verb|\citet*{smith2020}| produces: Smith, Jones, and Lee (2020)

\item \verb|\citep[p.~25]{key}| → ページ番号付き
  
  Example: \verb|\citep[p.~25]{smith2020}| produces: (Smith, 2020, p. 25)

\item \verb|\citep[see][]{key}| → 前置詞付き
  
  Example: \verb|\citep[see][]{smith2020}| produces: (see Smith, 2020)
\end{itemize}

\newpage
\section{Method 4: biblatex (最新・高機能)}

% プリアンブルに追加:
% \usepackage[style=authoryear,backend=biber]{biblatex}
% \addbibresource{references.bib}

% 引用コマンド
\subsection{biblatex Citation Commands}

\begin{verbatim}
\cite{key}           % 基本引用
\parencite{key}      % 括弧付き引用
\textcite{key}       % テキスト内引用(著者名)
\footcite{key}       % 脚注引用
\autocite{key}       % 自動選択
\citeauthor{key}     % 著者名のみ
\citeyear{key}       % 年のみ
\citetitle{key}      % タイトルのみ
\end{verbatim}

% 文末に:
% \printbibliography

\newpage
\section{主要な BibTeX エントリタイプ}

\begin{verbatim}
@article{...}        % 学術論文
@book{...}           % 書籍
@inbook{...}         % 書籍の章
@incollection{...}   % 論文集の中の章
@inproceedings{...}  % 会議論文
@conference{...}     % 会議(inproceedingsと同じ)
@proceedings{...}    % 会議論文集全体
@phdthesis{...}      % 博士論文
@mastersthesis{...}  % 修士論文
@techreport{...}     % 技術レポート
@manual{...}         % マニュアル
@misc{...}           % その他(Webページなど)
@unpublished{...}    % 未公刊
\end{verbatim}

\newpage
\section{主要な bibliographystyle オプション}

\subsection{標準スタイル}
\begin{itemize}
\item \texttt{plain} - 著者名のアルファベット順、番号付き
\item \texttt{alpha} - 著者名+年の略号([Smi20])
\item \texttt{unsrt} - 引用順、番号付き
\item \texttt{abbrv} - plain の短縮版
\end{itemize}

\subsection{natbib 用スタイル}
\begin{itemize}
\item \texttt{plainnat} - 著者名-年形式
\item \texttt{abbrvnat} - 短縮版
\item \texttt{apalike} - APA スタイル
\end{itemize}

\subsection{学会別スタイル}
\begin{itemize}
\item \texttt{ieeetr} - IEEE Transactions
\item \texttt{acm} - ACM
\item \texttt{apalike} - APA (心理学)
\item \texttt{chicago} - Chicago Manual
\end{itemize}

\newpage
\section{引用の実例}

\subsection{Example 1: 基本的な引用}
\begin{verbatim}
Deep learning has shown remarkable results~\cite{lecun2015}.
\end{verbatim}

\subsection{Example 2: 複数の引用}
\begin{verbatim}
Several studies~\cite{smith2020,jones2019,brown2021} have shown...
\end{verbatim}

\subsection{Example 3: ページ番号付き引用}
\begin{verbatim}
As stated in~\cite[p.~42]{smith2020}, the algorithm is efficient.
\end{verbatim}

\subsection{Example 4: 著者名を含む引用 (natbib)}
\begin{verbatim}
\citet{smith2020} proposed a novel method.
According to \citet{smith2020}, the approach is effective.
This was demonstrated by previous work~\citep{jones2019,brown2021}.
\end{verbatim}

\subsection{Example 5: 脚注引用 (biblatex)}
\begin{verbatim}
The theory was first proposed in 1950.\footcite{original1950}
\end{verbatim}

\section*{コンパイル手順}

\subsection*{BibTeX を使う場合}
\begin{verbatim}
pdflatex document.tex
bibtex document
pdflatex document.tex
pdflatex document.tex
\end{verbatim}

\subsection*{biblatex (biber) を使う場合}
\begin{verbatim}
pdflatex document.tex
biber document
pdflatex document.tex
pdflatex document.tex
\end{verbatim}

\end{document}