\documentclass[12pt,a4paper]{article}

% Package imports
\usepackage[utf8]{inputenc}
\usepackage[english]{babel}
\usepackage{amsmath,amssymb,amsthm}
\usepackage{graphicx}
\usepackage[margin=2.5cm]{geometry}
\usepackage{cite}
\usepackage{url}
\usepackage{natbib}
\usepackage{hyperref}

% Title information
\title{Paper Title}
\author{Author Name\\
        Affiliation\\
        \texttt{email@example.com}}
\date{\today}

% Theorem environments
\newtheorem{theorem}{Theorem}
\newtheorem{lemma}[theorem]{Lemma}
\newtheorem{proposition}[theorem]{Proposition}
\newtheorem{corollary}[theorem]{Corollary}
\theoremstyle{definition}
\newtheorem{definition}[theorem]{Definition}
\newtheorem{example}[theorem]{Example}
\theoremstyle{remark}
\newtheorem{remark}[theorem]{Remark}

\begin{document}

\maketitle

\begin{abstract}
Write your abstract here. Summarize the purpose, methods, main results, and conclusions of your research concisely. Typically 150-250 words.
\end{abstract}

\tableofcontents
\newpage

\section{Introduction}
\label{sec:introduction}

Present the background, motivation, and purpose of your research.

\subsection{Background}
Review existing research and related work.

\subsection{Research Objectives}
Clearly state the objectives and contributions of this research.

\section{Preliminaries}
\label{sec:preliminaries}

Present necessary definitions and known results.

\begin{definition}
Example definition: A function $f: X \to \mathbb{R}$ on a set $X$ is said to be continuous if...
\end{definition}

\begin{theorem}
\label{thm:main}
Statement of the main theorem.
\end{theorem}

\begin{proof}
Write the proof here.
\end{proof}

\section{Main Results}
\label{sec:main}

Present the main results of your research.

\subsection{Result 1}

Example of mathematical equations:
\begin{equation}
\label{eq:example}
E = mc^2
\end{equation}

Equation~\eqref{eq:example} is the famous formula from the theory of relativity.

\subsection{Result 2}

Example of figure insertion:
\begin{figure}[htbp]
\centering
% \includegraphics[width=0.6\textwidth]{figure1.pdf}
\caption{Figure caption}
\label{fig:example}
\end{figure}

As shown in Figure~\ref{fig:example}...

\section{Experiments and Numerical Examples}
\label{sec:experiments}

Present experimental results and numerical examples.

\begin{table}[htbp]
\centering
\caption{Example of experimental results}
\label{tab:results}
\begin{tabular}{|c|c|c|}
\hline
Item 1 & Item 2 & Item 3 \\
\hline
Data 1 & Data 2 & Data 3 \\
Data 4 & Data 5 & Data 6 \\
\hline
\end{tabular}
\end{table}

As can be seen from Table~\ref{tab:results}...

\section{Discussion}
\label{sec:discussion}

Discuss the implications of your results.

\section{Conclusion}
\label{sec:conclusion}

Summarize the research and describe future work.

\subsection{Summary}
In this research, we have...

\subsection{Future Work}
Future research directions include...

\section*{Acknowledgments}
This research was supported by...

\begin{thebibliography}{99}

\bibitem{author2020}
Author Name,
``Paper Title,''
\textit{Journal Name},
vol.~XX, no.~Y, pp.~ZZ--WW, 2020.

\bibitem{book2019}
Author Name,
\textit{Book Title},
Publisher, 2019.

\bibitem{web2021}
Author Name,
``Web Page Title,''
\url{https://example.com},
Accessed: January 1, 2021.

\end{thebibliography}

\end{document}